\subsection{Summary of the project and architecture}

This technical report has presented the preliminary architecture of an integrated modeling pipeline that joins together long-term land use, short-term travel demand, and static user equilibrium traffic assignment. Our land use model, UrbanSim, is an open-source microsimulation platform used by metropolitan planning organizations worldwide for modeling the growth and development of cities over long time horizons. Our travel demand model, ActivitySim, is an agent-based modeling platform that produces synthetic origin--destination travel demand data. Finally, we use a static user equilibrium traffic assignment model based on the Frank-Wolfe algorithm to assign vehicles to specific network paths to make trips between these origins and destinations. This traffic assignment model runs in a high-performance computing environment and the resulting congested travel time data can then be fed back into UrbanSim and ActivitySim for the next model run. 

This ongoing effort will focus on adding workplace choice and vehicle ownership into the UrbanSim models and tightly integrating this with network models. With the behaviorally integrated models and a sufficiently detailed representation of the transportation network and geography, ideally using local streets and parcels and buildings, we will be able to explore the long-term feedback effects of transportation infrastructure changes on urban development patterns. We will be able to include the feedback effects of these urban development dynamics on travel demand and on travel flows and speeds, and consequently on transportation-related energy consumption.

\subsection{Accomplishments to date}

In this preliminary phase of the project, we have recently completed an integration of our land use model (UrbanSim) and travel demand model (ActivitySim), which marks a first for urban microsimulation. We are constructing graph models of metropolitan-scale road networks, on-demand, with configurable resolution (that is, tertiary roads and up, or all drivable roads if desired). Finally, we recently completed a preliminary hand-off of our synthetic travel demand data to a static user equilibrium traffic assignment model in an HPC environment, with a successful model run.

This approach to integrated modeling offers several future benefits. It will allow for initial deliverables to be generated in the near term, e.g. a calendar year. The tools and analytics being used are robust and validated by previous research efforts and by operational applications among leading MPOs in the country. A community of users have already developed the necessary regional data for many urban environments, including the Bay Area, San Diego, Denver, Detroit, Seattle, etc. Finally, once developed, future efforts can begin to analyze more sophisticated policy scenarios

By creating the combined models with sufficient performance to simulate the feedback of land use and travel and congestion annually rather than every 5 to 10 years, a variety of alternative scenarios could be developed and explored that couple land use policies. Examples include: transit oriented development with transportation networks emphasizing transit, or more efficient highway projects coupled with supporting land use policies, with evaluation of the induced demand effects of both. Policies that affect parking construction and management, or which begin to explore the potential impacts of greater adoption of ride hailing services and eventually autonomous vehicles, could also be incorporated as data availability permits.

\subsection{Connections to energy use}
One objective of this work, in which we jointly consider long term land use, short term activity choices, and traffic assignment, is to observe the energy impacts that result as a consequence of particular decisions. Perhaps the most straightforward way to measure the energy impact of this system is to consider the emissions associated with transportation. Vehicular emissions can be calculated as a function of travel time \pcite{ahn2008effects}. This is useful even with a static traffic assignment model - in which the results are representative of travel time on each link and can therefore be used to calculate emissions costs \pcite{aziz2012integration}. As we move towards a dynamic traffic model, in which the results can be thought of as actual travel times and the effects of dynamic flow in the network are captured, these energy costs will become more precise.

\subsection{Open issues}
The primary challenge in traffic assignment involves the transition from static to dynamic. As described above, static user equilibrium is a well defined mathematical notion in which there is a closed form solution and numerous algorithms exist to compute this solution. On the other hand, although many dynamic flow models exist, there is no well defined notion of dynamic user equilibrium. Because of this, it is much less clear whether the results of a dynamic traffic assignment are behaviorally realistic. In order to validate the results from dynamic traffic assignment, these results must be compared with observed data. Although such a comparison can be used to gain an understanding of the performance of the dynamic model, there is no clear definition to understand exactly how well the dynamic assignment is performing. 

The main technical challenge still preventing a deeper integration of our three modeling components relates to the issue of accessibilities/composite utilities. These measures are fundamental to the destination choice models of both UrbanSim and ActivitySim, and must be updated to reflect the latest network impedances as computed during the traffic assignment step. UrbanSim is configured to estimate its own accessibilities directly from a transportation network file using the Pandana Python package. ActivitySim, however, currently expects both an updated skims matrix and an associated set of destination logsums as inputs for its accessibility model step. Because our travel model is not generating any composite utilities beyond travel times, we have no way of updating these inputs [\textit{need to check with Ben Stabler on this point}] to incorporate the latest traffic assignment results for simulations beyond the first iteration. As such, a high priority for us is to reconfigure ActivitySim and all of its sub-models to run on the same Pandana-derived accessibility measures as UrbanSim. This will not only reduce the number of accessibility computations by a factor of two, but it will also significantly reduce the amount of overhead required for post-processing the traffic assignment output.


\subsection{Future research agenda}

The upcoming research agenda for this project covers five primary areas: the network model, the travel demand model, traffic assignment, deeper integration of the pipeline, and benchmarking.

\subsubsection{Network model}

We will be improving the network model in two ways. It currently includes tertiary roads and higher, discarding local and residential roads both for performance as well as because these types of edges historically have been unimportant to automobile-focused traffic assignment and congestion modeling. We will produce an alternative network model that includes all drivable edges to compare traffic assignment performance in the HPC cluster. Also, our zone-based synthetic travel demand data is currently mapped to the center-most node in each zone, from which all zone trips originate and to which all zone trips end. We will instead distribute zone-based origins and destinations according to a probability distribution across all the nodes in a zone.

\subsubsection{Travel demand model}

Our research agenda includes improvements of the travel demand model, ActivitySim. In addition to addressing the issue of accessibility calculations described in the previous section, we will be moving long-range models (including school choice, automobile ownership choice, and workplace choice) to UrbanSim, a more natural home for long-term decision modeling. Furthermore, we will research run-time improvements for ActivitySim to reduce its currently substantial execution time. We will also be working to more tightly couple the UrbanSim and ActivitySim runtime environments, eventually executing models from both platforms from within a single Python-based data pipeline orchestration framework. 

To increase the behavioral realism, we will simplify the behavior in the ActivitySim model system developed using the same software core as UrbanSim, and integrate the resulting models more directly into UrbanSim as a simplified, integrated, light-weight Activity Based Travel Demand model system. This will then be tightly coupled to a distributed network flow model to simulate route choices, network loading and generate updates of congested travel times. Using a parallel implementation of the network assignment model, our goal is to speed up the traffic assignment step dramatically, enabling it to be coupled to UrbanSim plus activity based travel demand model components that will run every simulated year.

In order to leverage application by MPOs, the approach we propose is to add functionality for including longer-term choices such as workplace, school location, and vehicle ownership to UrbanSim. This will significantly simplify the computation in ActivitySim and will enable a clear and viable path to achieving a long-term coupled urban dynamic and transport modeling framework that can effectively inform energy modeling via the rapid creation and evaluation of alternative scenarios. 

\subsubsection{Traffic assignment}

We have completed a preliminary traffic assignment on our network model with our travel demand data. Next we will be modeling real-world scenarios using the San Francisco Bay Area land use data, travel demand data, and network model. Then we will be testing multiple scenarios, including networks with fewer or more edges and signal timing. Finally we will be testing different traffic models, including Merchant Nemhauser, cellular automata, and link delay.

Initially, stochastic user equilibrium assignment models will provide the network flow modeling. Following this we propose to explore more robust network flow algorithms based on decomposition of convex programs. The network flow algorithms are easily decomposed for deployment in parallel computing environments (e.g., cloud and HPC). The parallelization of these algorithms will allow the travel demand and consequent network flow models to run fast enough to enable rapid integrated modeling with the UrbanSim framework. This can be handled by accelerated Frank-Wolfe algorithms that can run in a decentralized way on HPC platforms. More specifically, run times for UrbanSim for one simulation year in a large region such as the Bay Area is on the order of a few minutes, while simulating one day with the regional travel model requires many hours. Our goal is to reduce the combined model system to a run time under one hour per simulated year.

\subsubsection{Pipeline integration}

The current pipeline demonstrates a \enquote{loosest coupling} methodology wherein data inputs and outputs are serialized to disk for hand-off to the other modeling steps. Because UrbanSim and ActivitySim are both built on top of the orca model orchestration pipeline, we have been able to integrate their inputs and outputs. However, our research agenda includes automating the hand-off to the traffic assignment model, and pipelining the traffic assignment model's outputs as inputs that feed back into UrbanSim and ActivitySim to provide congested travel time information for location and travel decision making.

\subsubsection{Validation and benchmarking}

Finally, our research agenda includes substantial benchmarking and validation of this modeling pipeline. The benchmarking will measure run-time for various configurations of the models, with a research objective of improving run-time performance particularly for the travel demand model. We will also be exploring trade-offs between model granularity and representation versus computational performance. The validation step will compare our model's outputs with real-world data to verify plausible results.

Several MPOs would be in a position to evaluate and potentially apply this new framework, including MTC in the Bay Area, SANDAG in San Diego, PSRC in Seattle, and potentially SEMCOG in Detroit and DRCOG in Denver, as well as smaller MPOs such as PPACOG in Pikes Peak and NFRMPO in Colorado Springs, and in OahuMPO in Honolulu -- all of whom are currently using UrbanSim in their operational long-term transportation planning processes. We will draw on this experience to validate our approach to creating this simplified lightweight travel demand generator that is more appropriate for the time scales being modeled in the UrbanSim framework. 

\subsection{Deliverables}
This project has four key deliverables due to the Department of Energy over FY18. Our Q1 deliverable is this technical report on the progress so far and the preliminary architecture of this integrated modeling pipeline. The Q2 deliverable is a network flow model running at scale, integrated with UrbanSim and ActivitySim. The Q3 deliverable is a journal article manuscript describing these accomplishments, to be submitted for peer review. Finally, the Q4 deliverable is a code repository for this code to run at scale.
