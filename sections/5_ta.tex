\subsection{Overview}

The traffic assignment component of this integrated modeling pipeline provides vehicles in the network specific paths to make trips between their origins and destinations (ODs). The origins, destinations, and numbers of vehicles making trips are received from ActivitySim and then assigned to specific network edges through traffic assignment. The resulting path assignment results in a total number of vehicles traveling on each edge in the transportation network, which is then used to compute a travel time on each edge. The travel time on an edge is a function of the number of vehicles using that edge. These travel times are then given back to UrbanSim and ActivitySim to be used for congested accessibility and skims.


\subsection{Inputs}

The primary inputs used by the traffic assignment model comprise the network (including the associated origin-destination travel demand) and BPR coefficients associated with each edge. These inputs were described in detail in Sections \ref{sec:activitysim} and \ref{sec:network} and reviewed briefly here.


\subsubsection{The network}

The network infrastructure is based on a set of nodes and edges in which connected edges share a node. Vehicles are allowed to travel from one edge to another if the edges are connected via a node. The ordered set of these edges (in which the ordering denotes a shared node between two edges) are called a path. In order for a vehicle to move from its origin to its destination, it must take a particular path. The possible paths between origins and destinations are defined by the topology of the network (i.e., the number of edges, the number of nodes, and the connections between these nodes and edges). The network described in Section \ref{sec:network} is used as the foundation of the traffic assignment model. 


\subsubsection{BPR coefficients}

The BPR coefficients computed previously are used to assign a pseudo travel time to each edge as a function of the edge's load (i.e. number of vehicles on the edge). This is necessary in order to determine which paths vehicles should be assigned to (since vehicles are more likely to take a path which requires less time).

\subsection{How it works}

The currently implemented version of the traffic assignment model determines a static user equilibrium using the Frank-Wolfe algorithm. Static traffic assignment does not consider time varying parameters of any kind so there is no concept of flow dynamics but there is a well-defined equilibrium. Static user equilibrium, often called Wardrop's first principle in the transportation literature, is a well-defined state in which all vehicles take the shortest path from their origin to their destination. The resulting traffic assignment is based on shortest paths that are calculated while considering the travel times resulting from a loaded network in which each user is attempting to minimize their travel time. 

Wardrop's first principle states that the actual travel time experienced by a user in the network is equal or less than the travel time that the same driver would experience on any other route \citep{wardrop1952road}. Static user equilibrium---which is equivalent to Nash equilibrium---is defined in Equation \ref{eq:static_user_equilibrium}:

\begin{equation}
    \label{eq:static_user_equilibrium}
    \begin{split}
    \min \sum_{e\in\ \textit{edges}}\int_{0}^{v_e} S_e(x) dx\\
    \text{subject to} \\ 
    v_e = \sum_i \sum_j \sum_r \alpha_{ij}^{er} x_{ij}^{r}\\
    \sum_r x_{ij}^{r} = T_{ij}\\
    v_e \geq 0 \\
    x_{ij}^{r} \geq 0
    \end{split}
\end{equation}

The minimization is over travel time on each edge as a function of traffic volume. $x_{ij}^{r}$ is the number of vehicles on path $r$ from node $i$ to node $j$. $\alpha_{ij}^{ar}$ is equal to 1 if edge $a$ is contained in path $r$ and 0 otherwise. To perform this static user equilibrium calculation we use the Frank-Wolfe algorithm \citep{frank1956algorithm}. The Frank-Wolfe algorithm is a first-order optimization algorithm that is used to solve convex problems such as the static user equilibrium problem defined above. The algorithm works as follows; consider a problem of the form:

\begin{align*}
    S \in \mathcal{R}^n \text{ a polyhedron and } f: \mathcal{R}^n \rightarrow \mathcal{R} C^1, \text{ solve:}
\end{align*}

\begin{align*}
    \text{min. } f(x) \text{ subject to } x \in S
\end{align*}

Frank-Wolfe algorithm:

\begin{enumerate}
    \item Initialize with $x_0 \in S$ and let $k= 0$
    \item Determine a search direction $d_k = y_k - x_k$ by solving the linear program: \\
    $y_k \in \text{argmin}_{y \in S} \{ \nabla f(x_k)^Ty \} $
    \item Determine a step length $\alpha_k \in [0.1]$ such that: \\
    $f(x_k + \alpha_k d_k) = f((1 - \alpha)x_k + \alpha y_k) \leq f(x_k)$
    \item Update $x_{k+1} = (1 - \alpha)x_k + \alpha y_k$ let $k = k+1$ and go to step 2
\end{enumerate}

\subsection{Outputs}

From the traffic assignment model, we obtain the number of vehicles on each edge in the network and the resulting travel time (calculated using the BPR coefficients) of each edge. Since these travel times are a result of a static traffic assignment, they are not interpretable as exact travel times and the units are effectively meaningless. Rather, they represent the congested travel time of each edge calculated using the BPR equations.

As described previously, these BPR function are empirical. These travel times are used to generate congested skims and accessibility, in which the congestion on each edge in the network is known in relative terms, and therefore areas of relatively high versus low congestion can be specified. UrbanSim and ActivitySim use these congested edge travel times to calculate accessibility and skims. For these calculations, the precise definition of travel time is not necessary; the comparison between edges is the important component.


\subsection{Calibration and validation}

In its current implementation, the traffic assignment model is static, meaning that back-propagation of congestion and other time dependent phenomena observed in real transportation networks cannot be well-modeled. However, the resulting travel times due to loading on each edge can be compared with real data to understand how well the static approximation matches with the observed travel times on a network. 
